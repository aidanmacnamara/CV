%%---------------------------------------------------------------------------%
%
% Notes:
%
% * Don't forget to change `pdfauthor' and `keywords' in the \hypersetup
%   section below.
%
% * To create a new page use: \newpage \opening
%
% * res.cls includes an \address{} command which can be used up to twice,
%   but my address is too long for the format it uses.
%
% * Alternate documentclass statement to put headings in margin:
%   \documentclass[margin,line,11pt,final]{res}
%
% * Can divide publication/presentation list into subsections by year:
%   \section{\bf\small\hspace{8mm}2006}
%
%%----------------------------------------------------------------------------%
\documentclass[overlapped, line, 11pt, letterpaper]{res}
%\documentclass[margin,line,12pt,letterpaper]{res}
%\textheight = 46\baselineskip
\newcommand{\comment}[2][]{{\bf #1}}
\newcommand{\superscript}[1]{\ensuremath{^\textrm{\scriptsize #1}} }

%\newcommand{\th}[0]{\superscript{th}}
\newcommand{\st}[0]{\superscript{st}}
\newcommand{\nd}[0]{\superscript{nd}}
\newcommand{\rd}[0]{\superscript{rd}}
\renewenvironment{itemize}{
\begin{list}{$\bullet$}
% inform the list command to use this counter
{\setlength{\itemsep}{0cm}}
}{\end{list}}


%\usepackage{res_yy}
\usepackage{ifpdf}
\usepackage{amssymb}
\usepackage{txfonts}
\ifpdf
	\usepackage[pdftex]{hyperref}
\else
	\usepackage[hypertex]{hyperref}
\fi

\hypersetup{
  letterpaper,
  colorlinks,
  urlcolor=black,
  pdfpagemode=none,
  pdftitle={Curriculum Vitae},
  pdfauthor={Aidam MacNamara},
  %pdfcreator={$ $Id: cv-us.tex,v 1.28 2006/12/12 22:53:52 jrblevin Exp $ $},
  pdfsubject={Curriculum Vitae},
  pdfkeywords={}
}

%%===========================================================================%%

\begin{document}

%---------------------------------------------------------------------------
% Document Specific Customizations

% Make lists without bullets and with no indentation
\setlength{\leftmargini}{1em}
\renewcommand{\labelitemi}{$\bullet$}
%\setlength{\leftmarginii}{1em}
%\renewcommand{\labelitemii}{--}


% Use large bold font for printed name at top of pages
\renewcommand{\namefont}{\large\textbf}

%---------------------------------------------------------------------------

\name{Aidan MacNamara}
%\address{The Mill House Wing\\Streetly End, West Wickham\\Cambridgeshire CB21 4RP\\}
\begin{resume}

\begin{ncolumn}{2}
  3 Park Lane		& Phone:+44 (0) 1799 585 001\\
  Castle Camps	& {\small \tt aidan.macnamara@gmail.com} \\
  Cambridgeshire		& Nationality: Irish \\
  CB21 4SS	&
\end{ncolumn}

%---------------------------------------------------------------------------

\section{\bf Summary}

Having gained a diverse experience in statistics and computational modeling through my PhD and post-doctoral positions, I now apply network and modeling approaches to target identification and validation in drug discovery as part of the Functional Genomics group at GSK. My goal is to establish such approaches as an integral component of drug discovery and demonstrate significant and successful results of their use. 

%---------------------------------------------------------------------------

\section{\bf Work Experience}
\begin{formatb}
\title{l} \dates{r}\\
\location{l}\\
\body \\
\end{formatb}

\title{\it Computational Biologist} \dates{Jan 2015 -- present}
\location{GSK, Stevenage}
\begin{position}
As a computational biologist within the Target Sciences group of GSK, I apply network and 'omics-integration approaches to diverse data types (genomic, transcriptomic and proteomic) to better understand potential drug targets. More specifically:
\begin{itemize}
\item I lead the analysis of an Open Targets project that uses data-driven multi-'omic approaches to find better cell models for disease \emph{(https://www.opentargets.org/)}.
\item I work closely with the immune/inflammation therapy area, applying network approaches to target discovery.
\item I have a strong focus on epigenetic target discovery - this consists of design and analysis of multi-'omic experiments to better understand potential target mechanisms.
\item I am interested in the use of tissue multiplexing techniques for target discovery and am working across several projects that are using and developing this technology.  
\end{itemize}
\end{position}

\title{\it Postdoctoral Fellow in Systems Biology} \dates{Nov 2010 -- Dec 2014}
\location{EMBL-EBI, Cambridge}
\begin{position}
I held a postdoctoral fellow position in the group of Julio Saez-Rodriguez (Systems Biomedicine). My research concentrated on using single-cell data and computational modeling to better understand the role of signal deregulation in disease. This includes the analysis and modeling of mass cytometry, fluorescent microscopy, proteomic and gene expression data types.
\end{position}

\section{\bf Education}
\begin{formatb}
\title{l} \dates{r}\\
\location{l}\\
\body \\
\end{formatb}

\title{\it PhD in Immunology} \dates{Oct 2006 -- Sep 2010}
\location{Imperial College, London}
\begin{position}
The determination of the basis of HLA class-I protection in HTLV-I infection. Using theoretical methods to understand the immune response to the retrovirus HTLV-I, I gained insight into how the host genotype protects against diseases associated with this virus. Funded by the Wellcome Trust and supervised by Dr. B. Asquith and Prof. C. Bangham.
\end{position}

\emph{MSc in Bioinformatics} \hfill Oct 2005 -- Sep 2006\\
Imperial College, London
 
\emph{BA (2:1) in Zoology} \hfill Oct 1998 -- Jun 2002\\
Trinity College, Dublin

\section{Publications}
%\setlength{\leftmargini}{0em}
%\renewcommand{\labelitemi}{}
\textbf{\emph{First Author}}
\begin{itemize}
\item An assessment of genetic proxies: how networks and mechanisms lead to alternative drug targets. \emph{In preparation}.
\item Epigenetic and transcriptional regulation of monocyte differentiation: a comparison of primary and immortalised cell models. \emph{In preparation}.
\item A single-cell model of PIP3 dynamics using chemical dimerization. MacNamara A, Stein F, Feng S, Schultz C, and Saez-Rodriguez J. \emph{Bioorganic \& Medicinal Chemistry}, 23(12), 2868-2876, 2015.
\item Modeling signaling networks with different formalisms: a preview. MacNamara A, Henriques D and Saez-Rodriguez J. \emph{Methods in Molecular Biology}, 1021, 89–105, 2013.
\item State-time spectrum of signal transduction logic models. MacNamara A, Terfve CDA, Henriques D, Pen\~alver Bernab\'e B and Saez-Rodriguez J. \emph{Physical Biology}, 9, 4, 45003, 2012.
\item HLA class I binding of HBZ determines outcome in HTLV-1 infection. MacNamara A, Rowan AG, Hilburn S, Kadolsky U, Fujiwara H, Suemori K, Yasukawa M, Taylor GP, Bangham CRM and Asquith B. \emph{PLoS pathogens}, 4, 9, e1001117, 2010.
\item The avidity and lytic efficiency of the CTL response to HTLV-I. Kattan T*, MacNamara A*, Rowan AG, Nose H, Mosley AJ, Tanaka Y, Taylor GP, Asquith B and Bangham CRM. \emph{The Journal of Immunology}, 182, 5723-5729, 2009.
\item T-cell epitope prediction: rescaling can mask biological variation between MHC molecules. MacNamara A, Kadolsky U, Bangham CRM and Asquith B. \emph{PLoS Computational Biology}, 5, e1000327, 2009.
\end{itemize}

\textbf{\emph{Others}}
\begin{itemize}
\item Modeling Signaling Networks to Advance New Cancer Therapies. Saez-Rodriguez J, MacNamara A and Cook S. \emph{Annual Review of Biomedical Engineering}, 17, 143-163, 2015.
\item A rapidly reversible chemical dimerizer system. Feng S, Laketa V, Stein F, Rutkowska A, MacNamara A, Depner, S, Klingm\"uller U, Saez-Rodriguez J and Schultz C. \emph{Angewandte Chemie}, 2014.
\item PIP3 Induces the Recycling of Receptor Tyrosine Kinases. Laketa V, Zarbakhsh S, Traynor-Kaplan A, MacNamara A, Subramanian D, Putyrski M, Mueller R, Nadler A, Mentel M, Saez-Rodriguez J, Pepperkok R, and Schultz C. \emph{Science Signaling}, 7 (308), 2014.
\item Frequency and function of KIR+ CD8+ T cells in HTLV-1 infection. Twigger K, Rowan A, Seich al Basatena N, MacNamara A, Retiere C, Gould K, Taylor GP, Asquith B and Bangham CRM. \emph{Retrovirology}, 11, 79, 2014.
\item MEIGO: an open-source software suite based on metaheuristics for global optimization in systems biology and bioinformatics. Egea JA, Henriques D, Cokelear T, Villaverde AF, MacNamara A, Danciu P, Banga JR and Saez-Rodriguez J. \emph{BMC Bioinformatics}, 15(1), p.136, 2014.
\item In contrast to HIV, KIR3DS1 does not influence outcome in HTLV-1 retroviral infection. O'Connor GM, Seich Al Basatena N, Olavarria V, MacNamara A, Vine A, Ying Q, Hisada M, Galv\~ao-Castro B, Asquith B and McVicar DW. \emph{Human Immunology}, 2012.
\item CellNOptR: a flexible toolkit to train protein signaling networks to data using multiple logic formalisms. Terfve CDA, Cokelear T, Henriques D, MacNamara A, Gon\c{c}alves E, Morris, MK, Van Iersel M, Lauffenburger DA and Saez-Rodriguez J. \emph{BMC Systems Biology}, 6, 133, 2012.
\item An IFN-$\gamma$ ELISPOT assay with two CTL epitopes derived from HTLV-1 Tax region 161-233 discriminates HAM/TSP patients from asymptomatic HTLV-1 carriers in a Peruvian population. Best \emph{et al.}, (3\textsuperscript{rd} author). \emph{AIDS research and human retroviruses}, 2011.
\item KIR2DL2 enhances protective and detrimental HLA class I-mediated immunity in chronic viral infection. Seich al Basatena \emph{et al.}, (2\textsuperscript{nd} author). \emph{PLoS Pathogens}, 7, 10, e1002270, 2011.
\item In vivo expression of human T-lymphotropic virus type 1 basic leucine-zipper protein generates specific CD8+ and CD4+ T-lymphocyte responses that correlate with clinical outcome. Hilburn S, Rowan AG, Demontis M, MacNamara A, Asquith B, Bangham CRM and Taylor GP. \emph{Journal of Infectious Diseases} 203, 4, 529, 2011.
\end{itemize}

\textbf{\emph{Software}}
\begin{itemize}
\item CNORdt - \emph{www.bioconductor.org/packages/release/bioc/html/CNORdt.html}. This is a method to train dynamic Boolean logic models to data.
\item CellNOpt - \emph{www.cellnopt.org}. A suite of methods to which I have significantly contributed that train logic models to data.
\item MEIGOR - \emph{www.iim.csic.es/~gingproc/meigo.html}. A toolbox that performs parameter estimation. I contributed a Bayesian inference method to this project.
\item epiChoose - \emph{http://epichoose.shiny.opentargets.io/epiChoose/}. A Shiny app that aids in the choice of cell models for primaty cells.
\end{itemize}

\section{\bf Teaching}
\textbf{\emph{Advisor}}
\begin{itemize}
\item I have supervised multiple Masters/internship research projects from the Universities of Cambridge and Warwick. I currenty co-supervise a PhD student as part of the Epimac network.
\end{itemize}
\textbf{\emph{Lecturing}}
\begin{itemize}
\item I have lectured at the FEBS/Wellcome Trust Practical Course ``In Silico Systems Biology'', EMBL-EBI, UK (2011-2013).
\item Visiting lecturer at EMBL-Heidelberg  (``Modeling of Network Signaling''), and Cranfield University (``Data integration and networks in drug discovery'').
\end{itemize}

\section{Awards}
The Wellcome Trust 4-year scholarship in bioinformatics (2005). \\
The EMBL EIPOD fellowship (2010).

\section{Skills}
\textbf{\emph{Computational}}
\begin{itemize}
\item \emph{Programming:} R, Python, UNIX, HPC 
\item \emph{Methods:} Network Biology, Multi-omics Integration, Machine Learning,  
\end{itemize}

\section{References}
Available upon request

\end{resume}

\end{document}

%%===========================================================================%%

